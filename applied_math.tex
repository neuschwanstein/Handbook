\part{Applied Mathematics, Optimization and Risk Theory}

\section{Convexity and Risk Theory}
\begin{outline}
\1 The \textbf{marginal utility} of utility $u$ is defined by $u''$. Therefore, a
decreasing marginal utility implies a risk-averse utility.
\reff{Eeckhoudt, p.9}

\1 \textbf{Jensen's Inequality.} If $g$ is a convex function and $X$ is a
random variable, then
\begin{equation*}
  g(E[X]) \leq E[g(X)].
\end{equation*}
Conversely, if $g$ is concave, then
\begin{equation*}
  g(E[X]) \geq E[g(X)].
\end{equation*}

\1 The \textbf{Taylor expansion} of $E[f(X)]$ is given by:
\begin{equation*}
  E[f(X)] = f(E[X])+ \tfrac{1}{2}\Var[X]f''(E[X]) + \cdots + \tfrac{1}{j!}\gamma_jf^{(j)}(E[X])
  + \cdots
\end{equation*},
where
\begin{equation*}
  \gamma_n = E[(X-E[X])^n].
\end{equation*}

% \1 The \textbf{Taylor expansion} of a function around the $E[X]$:
% \begin{equation*}
%   f(x) = f(E[X]) + \tfrac{1}{2}\Var(X)u''(\xi(x)).
% \end{equation*}

\1 If $g$ is convex and $g(0)\leq0$, then $g$ is \textbf{superadditive}, ie.
\begin{equation*}
  g(a) + g(b) \leq g(a+b).
\end{equation*}
Conversely, if $g$ is concave and $g(0)\geq 0$ then $g$ is \textbf{subadditive}, ie. 
\begin{equation*}
  g(a) + g(b) \geq g(a+b).
\end{equation*}

\1 The \textbf{certainty equivalent} of a lottery $\tilde x$ with utility $u$ is defined by
\begin{equation*}
  CE(\tilde x) = u^{-1}(Eu(\tilde x)).
\end{equation*}

\end{outline}

%%% Local Variables:
%%% mode: latex
%%% TeX-master: "handbook"
%%% End:
